\chapter{Zoeken in strings}
Dit hoofdstuk geeft uitleg over verschillende methodes om strings op te zoeken in een tekst.
Hierbij veronderstellen we dat we alle locatie willen vinden waar de string voorkomt. Merk op dat een string met zichzelf kan overlappen.
\npar
Daarvoor maken we gebruik van volgende terminologie:
\begin{itemize}
\item het patroon P dat we zoeken lengte = m 
\item tekst T met lengte = n
\item alfabet $\sum$ met d karaters 
\end{itemize}

\section{Variabele tekst}
\subsection{Eenvoudige methode}
\mijnfiguur[ht]{width=0.9\textwidth}{stringeen}{}
Deze methode is de eenvoudigste en zoekt gewoon of P vanaf een positie j voorkomt in T door de overeenkomstige karakters (op positie i) te vergelijken (P[i] == T[j+i-1] $\rightarrow$ i++; ). Indien er een verschillend karakter gevonden wordt zullen we de start positie eentje opschuiven (j++;) en opnieuw de karakters van P vergelijken.
\npar
Als de strings regelijk random zijn en we een groot alfabet hebben zal de eerste letter vaak verschillen waardoor we meestal na 1 vergelijking al kunnen opschuiven. Het gemiddeld geval is dan ook O(n).
\npar het slechtste geval is O(nm)dit doet zich voor als de eerste O(m) karakters van P op O(n) posities overeenkomen.(bv aaaab zoeken in aaaa....aaaaab) Dit lijkt zeer onwaarschijnlijk, maar als we de technieke gebruiken voor binaire strings of DNA strings met een klein alfabet is dit wel problematisch.

\subsection{Gebruik van de Prefixfunctie}
\subsubsection{Prefixfunctie}
\begin{table}[h]
\begin{tabular}{lrrrrrrrrr}
\textbf{index} &1 &2 &3 &4 &5 &6 &7 &8 &9 \\
\textbf{str} &a &a &b &a &a &b &a &b &b \\
                  \\
\textbf{p()} &0 &1 &0 &1 &2 &3 &4 &0 &0
\end{tabular}
\end{table}
De prefix functie p() van een string P bepaalt voor elke stringpositie i de lengte van het langste prefix van P, die overeenkomst met een deelstring van P, eindigend bij de positie i (1 <= i <= m). p(i) is dus steeds kleiner dan i en p(1) = nul.

\begin{lstlisting}
vector<int> PrefixFunction(string S) {
    vector<int> p(S.size());
    int j = 0;
    for (int i = 1; i < (int)S.size(); i++) {
        while (j > 0 && S[j] != S[i])
            j = p[j-1];

        if (S[j] == S[i])
            j++;
        p[i] = j;
    }   
    return p;
}
\end{lstlisting}
\clearpage
\subsubsection{Een eenvoudige lineaire methode}
\mijnfiguur[ht]{width=0.85\textwidth}{prefix}{}
Zoals je kan zien op \ref{prefix} zien we dat we het gevonden stuk string kunnen hergebruiken. We hebben namelijk ABABA al gevonden, echter de C klopt niet. Door nu echter de AB door te schuiven vermijden we extra vergelijkingen die toch negatief zouden zijn.
\npar
We berekenen m prefix waarden wat $\theta(m)$ tijd inneemt (gemiddeld). Dit betekent dat de buitenste herhaling gemiddeld (geatmortiseerd) O(1) is. Om dat aan te tonen moeten we gebruik maken van een potentiaal functie, de potentiaal komt overeen met de prefixfunctie zelf. 
Deze vormt zeker een bovengrens voor de uitvoeringstijd, want ze is initieel nul, en wordt nooit negatief. In de buitenste herhaling wordt ze minstens een kleiner (want we schuiven een op (zie 2 in de figuur)).
\npar
De afname van de potentiaal in de binnenste herhaling is dus minstens zo groot als het werk in die herhaling. De geatmortiseerde tijd van een buitenste herhaling is haar echte tijd plus de toename in potentiaal en dus 0(1). Aangezien er m-1 buitenste herhalingen zijn, is de totale performantie inderdaad $\theta(n)$.
\clearpage
\subsubsection{Knurth-Morris-Pratt}
We doen hetzelfde als hierboven maar met wat gepruts kunnen we het aantal binnenste iteraties verder beperken.
\begin{enumerate}
\item bepaal de prefix functie
\begin{tabular}{lrrrrrrrr}
char: &a &b &a &b &a &b &c &a \\
index:&0 &1 &2 &3 &4 &5 &6 &7 \\
p():  &0 &0 &1 &2 &3 &4 &0 &1
\end{tabular}
\item We overlopen de string tot we een fout vinden op positie i (in P).
\item i == 1 ? $\rightarrow$ p eentje naar rechts
\item i $>$ 1  ? $\rightarrow$ een prefix vinden in P met lengte i-1 die eindigt op positie j-1 in T. Nu moeten we verschuiven volgende de prefix functie
\item als P[ p(i-1) +1 ] == T[j] : na verschuiving komen deze letters overeen en kunnen we dus vanaf hier verder doen
\item als dat niet overeenkomst moeten we P nog wat verder opschuiven tot P.firstChar terug gelijk is aan T[i]
\item tot hier passen we eigenlijk gewoon de lineaire methode toe

\item \textbf{als P(i) != T[j] hebben we direct terug een fout;} we hadden dus beter de 2e langste prefix genomen ipv de langste.
\begin{itemize}
\item we hebben dus een bijkomende vereiste voor de prefix functie, namelijk dat het karakter rechts van de prefix verschilt van P[i]. Hierdoor zal de verschuiving groter zijn dan die van de originele methode. Dat het karakter verschilt kan niet garanderen dat het overkomt met T[j] (maar als het hetzelfde is dan is het zowieso fout dus), daarom moeten we eventueel nog een kleinere prefix nemen.
\item aangezien deze vereiste enkel afhankelijk is van P kunnen we ze op voorhand volledig bepalen, dit noemen we p2() of p'().
\end{itemize}
\clearpage
\item \textbf{Uiteindelijk gebruiken we nog een foutfunctie f(i)}
\begin{itemize}
\item (i) = p'(i-1) + 1
\item Deze functie geeft voor elke patroonpositie i groter dan een meteen de patroon positie waarmee T[j] moet vergeleken worden.
\item als we ook f(m+1) op die manier defini\"eren dan geeft die de nieuwe te testen patroon positie na het vinden van P. (Dus als we P vinden hoeveel moeten we dan schuiven om een volgende te vinden (overlap)).
\item Als i == 1 moet P een positie opgeschoven worden en wordt P[1] vergeleken met T[j+1] daarom moet f(1) = 1 zijn en p'(0) = 0 ; p'(1)  = 0
\end{itemize}
\end{enumerate}

het aantal karakter vergelijkingen van dit algoritme is $\theta(n)$. Want na elke verschuiving van P wordt hoogstens een karakter van T getest dat vroeger reeds getest werd. (Als het opnieuw niet overeenkomt wordt er doorgeschoven). Het totaal aantal karaktervergelijking is dus hoogstens gelijk aan de lengte T, plus het aantal verschuivingen. Elke verschuiving gebeurd over minstens 1 positie, zodat het aantal verschuivingen O(n) is. Dit levert dus idd $\theta(n)$.
\npar
Aangezien de voorbereiding van de prefix functie, de nieuwe prefixfunctie en de foutfunctie $\theta(m)$ is, wordt de totale performantie $\theta(m+n)$
\clearpage
\subsection{Karp-Rabin}
\mijnfiguur[ht]{width=0.9\textwidth}{krbasics}{}
\begin{itemize}
\item Indien getallen vergelijken sneller gaat dan strings, de stings omzetten naar getallen (met hashing) en die vergelijken.
\item als we dit doen krijgen we $d^{m}$ verschillende strings met lengte m, en dus evenveel getallen $\rightarrow$ te groot $\rightarrow$ delen door priemgetal $\rightarrow$ conflicten (zoals bij hashing). 
\item De getallen moeten in een processor woord passen!
\item gelijke strings $\rightarrow$ gelijke getallen maar gelijke getallen != gelijke strings
\item Op lossen door meerdere hashes te gebruiken of een volledige string vergelijking uit te voeren bij gelijke getallen

\item Hoe bekomen we deze getallen? 
\begin{itemize}
\item waarde deelstring op j+1 halen uit de waarde op j en dat in O(1)
\item Hiervoor beschouwen we een d tallig stelsel, waarbij elk stringelement voorgesteld wordt door een cijfer tussen 0 en d-1
\item H(p) = $\sum\limits_{i=1}^n$ dit vereist slechts m optellingen en evenveel vermenigvuldigingen. 
\item om beperkte hashwaarde te bekomen, nemen we de rest. 
\item doordat de rest van een som gelijk is aan de som van de resten van de termen, mogen we verder rekenen met de resten.
\end{itemize}
\item bewijs p103
\item perfomantie: $\theta(n+m)$
\item hoe p kiezen?
\begin{itemize}
\item p vast : hierbij nemen we een zo groot mogelijk priemgetal zodat $pd <= 2 ^w$.
Zo groot mogelijk zodat we minder valse positieven hebben, niet groter dan processor woord voor de effici\"entie.
\item p random: een of meerdere random p waarden gebruiken uit een interval.
\end{itemize}
\item Valse positieven?
\begin{itemize}
\item groter priemgetal nemen, maar let op processorwoord lengte
\item Omvormen naar een Las-Vegas algoritme door nog een string vergelijking uit te voeren bij een positieven getallen vergelijking
\item Herbeginnen met een nieuwe random p als dezelfde test fouten genereert.
\item meerdere fingerprints/hashes tegelijk gebruiken
\end{itemize}
\item tweedimensionale patroonherkenning of random vormpjes mogelijk.
\item tegelijk naar meerdere strings zoeken door verschillende hashwaarden te zoeken
\end{itemize}

\subsection{Shift-And / Shift-Or}
\begin{itemize}
\item eenvoudig, bitgeori\"enteerde methode die zeer effici\"ent werkt voor kleine patronen.
\item maakt gebruik van dynamisch programmeren
\item voor elke positie j in de tekst T worden de prefixen van P bijgehouden die eindigen op positie j. Dit zit in een eendimensionale tabel van m logische waarden.
\item het $i^{de}$ element komt overeen met de prefix van lengte i.
\item $R_{j}$ is de waarde die hoort die positie j, dan is $R_{j}$[i] waar als de eerste i karakters van p overeenkomen met de i tekstkarakters eindigt in j. Als P eindigt op j en $R_{j}$[m] waar is , dan hebben we P gevonden op plaats j-m+1
\item T[J+1] kan sommige prefixen verlengen tot aan j+1; $R_{j+1}$ kan dus uit $R_{j}$ gehaald worden als volgt:
\begin{itemize}
\item $R_{j+1}$[1] = ( P[1] == T[j+1] )
\item $R_{j+1}$[i] = ( $R_{j}$[i-1] \&\& P[i] == T[j+1] ) voor 1 $<$ i $<=$ m
\end{itemize}
\item Performantie: $\theta(nm)$ (wordt hieronder nog verder verbeterd)
\item bewerkingen zijn voor elk tabelelement analoog en we kunnen op meerdere bits tegelijk werken in het processorwoord.
\item $R_{j+1}$ is afhankelijk van T[j+1] == P[i]
\item P is vast dus op voorhand de posities bepalen waarvoor T[j+1] == [i] ; Dit doen we door een tabel S op te stellen met d woorden (met d = grootte van het alfabet).
\mijnfiguur[ht]{width=0.9\textwidth}{tabelS}{}
\item eenmaal we S hebben , moeten we niet meer vergelijken, maar kunnen we rechtstreeks in de tabel opzoeken of het waar of false is.
\item $R_{j+1} = Schuif(R_{j}) EN S[T[j+1]]$ Hierbij veronderstellen we dat er vooraan een waar bit (1) wordt ingeschoven. De waarde $R_{j+1}$ wordt dan volledig bepaald door de eerste bit van S[T[j+1]], maw ze is dus : ( T[j+1] == P[1]).
\item Deze operaties kunnen elk met 1 processor instructie gebeuren : O(1)
\item $R_{1}$ heeft geen voorloper $\rightarrow$ beginnen met een tabel die volledig onwaar is.
\item Shift-or : nul inschuiven; schift-and: 1 inschuiven
\item totale performantie is $\theta(n+m)$ ; zeer snelle methode voor kleine strings die bovendien weinig geheugen gebruikt.
\end{itemize}
\clearpage
\subsection{Boyer-Moore}
Boyer-Moore is nog performanter dan bovenstaande methodes. Het zoekt naar een patroon door het patroon van links naar rechts te schuiven over de tekst. \\
\mijnfiguur[ht]{width=0.9\textwidth}{BMShiftDir}{}

Een belangrijk verschil met de voorgaande methodes is dat we het patroon zelf van rechts naar links aflopen.\\
\mijnfiguur[ht]{width=0.9\textwidth}{BMCompModel}{}

boyer-moore maakt gebruik van 2 heuristieken om de performantie te garanderen.
\clearpage
\subsubsection{Verkeerde karakter}
Als we gestopt zijn bij een mismatch, dan kunnen we P naar rechts schuiven. Dit heeft echter slechts zin als het laatste character van P overeenkomt met het bovenliggende char in T (anders direct terug een fout). Je kan zien in de figuur dat we zullen doorschuiven tot waar B overeenkomt. Immers het stuk van C tot aan B in het pattern bevat geen eerdere B, en kan dus ook geen matches veroorzaken.\\
\mijnfiguur[ht]{width=0.9\textwidth}{BMBadChar1}{}

Een ander geval is dat er geen B in het Pattern zit, dan kunnen we direct het hele patroon doorschuiven.
\mijnfiguur[ht]{width=0.9\textwidth}{BMBadChar2}{}
\clearpage
Er bestaan echter verschillende varianten/implementaties van deze heuristiek. Merk op dat als we de heuristiek fout implementeren we foutieve verschuivingen kunnen krijgen in de omgekeerde richten. Als we bij \ref{BMBadChar1} bijvoorbeeld een B hadden rechts in het patroon zou men het patroon terug schuiven, wat uiteraard niet de bedoeling is.
\begin{enumerate}
\item \textbf{Uitgebreide heuristiek:} Hierbij vermeden we negatieve verschuivingen, door de meest rechtste P te nemen, links van de patroonpositie. Hierdoor is de verschuiving steeds positief
\item We houden een 2-dim tabel bij om de verschuiving voor elke mogelijk letter op elke mogelijk positie te bepalen. Bepalen door P eenmaal van links naar rechts te overlopen
\item Compromis tussen plaats en tijd:
\begin{itemize}
\item geen tabel maar een gelinkte lijst bijhouden die dalend gesorteerd is.
\item iets trager, maar gebruikt minder geheugen.
\end{itemize}

\item \textbf{Variant van Horspool:} betere oplossing die dezelfde tabel als de originele methode gebruikt. Voor elk karakter in het alfabet bevat de tabel de meest rechtse positie j van dat karakter in P , links van positie m. Als het karakter daar niet voorkomt dan is j = 0. Wanneer er bij het vergelijken een fout optreedt bij patroonpositie i en tekstpositie k dan moet P opgeschoven worden. Hierbij moet het nieuwe karakter terug gelijk zijn, dit kunnen we vinden met index T[k+m-i]. Vervolgens verschuiven we m-j positief, wat steeds positief is.
\npar
De verschuiving is bovendien onafhankelijk van de foutieve patroon positie i , en meestal groter dan de oorspronkelijke versie.

\item \textbf{Variant van Sunday:} gebruikt dezelfde tabel als de originele methode, maar als er een fout optreed tussen patroon positie i en tekstpositie k, dan zorgt men ervoor dat bij de verschuiving het gepaste patroonkarakter tegenover tekstkarakter T[k+m-i+1] terecht komt (net voorbij P[m] dus).
\npar
Hierbij is ook de verschuiving onafhankelijk van de foutieve patroonpositie i. Het gevolg hiervan is dat de karakters van P die met T vergeleken moeten worden, geen rol meer speelt. We kunnen ze dus overlopen op de kans dat ze voorkomen.
\end{enumerate}